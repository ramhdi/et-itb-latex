\chapter*{Kata Pengantar}
\addcontentsline{toc}{chapter}{KATA PENGANTAR}

Gunakan bagian ini untuk memberikan ucapan terima kasih kepada semua pihak yang secara langsung atau tidak langsung membantu penyelesaian tugas akhir, termasuk pemberi beasiswa jika ada. Utamakan untuk memberikan ucapan terima kasih kepada tim pembimbing tugas akhir dan staf pengajar atau pihak program studi, bahkan sebelum mengucapkan terima kasih kepada keluarga. Ucapan terima kasih sebaiknya bukan hanya menyebutkan nama orang saja, tetapi juga memberikan penjelasan bagaimana bentuk bantuan/dukungan yang diberikan. Gunakan bahasa yang baik dan sopan serta memberikan kesan yang enak untuk dibaca. Sebagai contoh: “Tidak lupa saya ucapkan terima kasih kepada teman dekat saya, Tito, yang sejak satu tahun terakhir ini selalu memberikan semangat dan mengingatkan saya apabila lengah dalam mengerjakan Tugas Akhir ini. Tito juga banyak membantu mengoreksi format dan layout tulisan. Apresiasi saya sampaikan kepada pemberi beasiswa, Yayasan Beasiswa, yang telah memberikan bantuan dana kuliah dan biaya hidup selama dua tahun. Bantuan dana tersebut sangat membantu saya untuk dapat lebih fokus dalam menyelesaikan pendidikan saya. ....”. Ucapan permintaan maaf karena kekurangsempurnaan hasil Tugas Akhir tidak perlu ditulis.
\\[\baselineskip]

Bandung, \thedate \\[\baselineskip]
Penulis